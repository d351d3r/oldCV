%%%%%%%%%%%%%%%%%%%%%%%%%%%%%%%%%%%%%%%%%
% Twenty Seconds Resume/CV
% LaTeX Template
% Version 1.0 (14/7/16)
%
% Original author:
% Carmine Spagnuolo (cspagnuolo@unisa.it) with major modifications by 
% Vel (vel@LaTeXTemplates.com), Harsh (harsh.gadgil@gmail.com) and johayon
%
% License:
% The MIT License (see included LICENSE file)
%
%%%%%%%%%%%%%%%%%%%%%%%%%%%%%%%%%%%%%%%%%

%----------------------------------------------------------------------------------------
%	PACKAGES AND OTHER DOCUMENT CONFIGURATIONS
%----------------------------------------------------------------------------------------
\documentclass[letterpaper]{twentysecondcvfr} % a4paper for A4

%\usepackage[utf8]{inputenc}
%\usepackage[T1]{fontenc}
%\usepackage[francais]{babel}

% Command for printing skill overview bubbles
\newcommand\skills{ 
~ 
	\smartdiagram[bubble diagram]{
        \textbf{Machine}\\\textbf{Learning},
        \textbf{Visualization},
				\textbf{~~~Deep~~~}\\\textbf{Learning},
        \textbf{~~~~~~~~Big~~~~~~~~}\\\textbf{Data},
        \textbf{Statistical}\\\textbf{Analysis},
				\textbf{Data}\\\textbf{Wrangling}
    }
}

% Programming skill bars
%\programming{ { Java $\textbullet$ MatLab $\textbullet$ Maple/ 3}, {Scala $\textbullet$ R %$\textbullet$ Spark $\textbullet$ \LaTeX/ 4}, {Python $\textbullet$ SQL/ 5}}

\tools{
\tikzset{
  every shadow/.style={
    fill=none,
    shadow xshift=0pt,
    shadow yshift=0pt}
}
\definecolor{pblue}{HTML}{0395DE}
\hspace{-1cm}
\smartdiagramset{set color list = {pblue,pblue,pblue,pblue,pblue,pblue,pblue}}
\smartdiagram[descriptive diagram]{
{\large \faFlask,{Sklearn Xgboost Caret RandomForest TensorFlow Keras NLTK Gensim Forecast Pandas Numpy}},
{\large \mfHadoop, {Spark MLlib GraphX Sparksql Zeppelin Livy}}, 
{\large \faDatabase,{MySql Elastic-Search Hbase Cassandra}}, {\large \faAreaChart,{Matplotlib Seaborn ggplot Plotly D3js Tableau \LaTeX}},
{\large \faCogs,{git Travis AWS Unittest Flask Airflow Vertx}}}
}

\programmings{23/materialteal/\textbf{\textbf{\textsf{R}}},  23/materialcyan/\textbf{\mfScala Scala}, 35/orange/\textbf{\ \mfPython Python}, 10.5/green/\textbf{\ \mfJavaBold Java}, 8.5/materialorange/\textbf{C++}}{2.25}{0.75}

% Interest icons text
\interests{ \textcolor{pblue}{\large \faBook \ \ \faMusic \ \ \faTv \ \ \faBicycle  \ \ \faGamepad \ \ \faAndroid \ \ \faLinux
}}

%----------------------------------------------------------------------------------------
%	 PERSONAL INFORMATION
%----------------------------------------------------------------------------------------
% If you don't need one or more of the below, just remove the content leaving the command, e.g. \cvnumberphone{}

\cvname{Jonathan Ohayon, Ph.D} % Your name
\cvjobtitle{ Data Scientist \\ Machine Learning } % Job
% title/career

\cvbirthday{12/04/1986} \cvnatio{Fran\c cais-Canadien} % Personal website
\cvhome{Versailles}  \cvnumberphone{+33 640958173} % Phone number
\cvmail{johayon.math@gmail.com} % Email address
\cvlinkedin{/in/johayonmath}
\cvgithub{johayon}

%----------------------------------------------------------------------------------------

\begin{document}

\makeprofile % Print the sidebar


%----------------------------------------------------------------------------------------
%	 EXPERIENCE
%----------------------------------------------------------------------------------------

\section{Exp\'erience}{\faAlignJustify}

\begin{twenty} % Environment for a list with descriptions
	\twentyitem
    	{2018 -}
	{Pr\'esent}
         {Data Scientist R\&D}
         {Air Liquide}
        {}
        {
				\begin{itemize}
				\item R\'ealisation et veille d'algorithme pr\'edictif dans la sant\'e.
				\item \'Evaluation des projets Marketing.
				\item \textbf{outils}: Python, R,  git
				\end{itemize} }\\

	\twentyitem
    	{2016 -}
		{Pr\'esent}
        {Data Scientist - Machine Learning Engineer}
        {FreeLance}
        {}
        {\begin{itemize}
        \item D\'eveloppement d'algorithmes de scoring.
        \item Mod\`eles de scoring pour des campagnes d'emails. 
        \item Analyses statistiques et mod\`eles pr\'edictifs dans la sant\'e. 
				\item \textbf{outils}: Python, R, jupyter, git, Unittest, Plotly
    \end{itemize}}\\
		
\twentyitem
    	{2016 -}
		{2018}
        {Data Scientist}
        {\href{http://www.holimetrix.ccom/}{Holimetrix}}
        {}
        {
				\begin{itemize}
				\item Attribution t\'el\'e \`a partir des traffics clients. 
				\item Cross-Device Pairing \`a travers tous les sites clients.
				\item Cr\'eation des sessions utilisateurs sur des T\'era de logs.
				\item \textbf{outils}: Python, Scala, Spark, GraphX, AWS, Travis, git
				\end{itemize}} \\

    \twentyitem
   		{2015 - 2016}
		{}
        {Data Scientist - Machine Learning Engineer}
        {\href{http://www.cetadata.com/}{CetaData}}
        {}
        {\begin{itemize}
				\item D\'eveloppement d'un mod\`ele de scoring en mode saas.
        \item Mod\`ele pr\'edictif des limitations de vitesse sur les routes.
				\item Segmentation des utilisateurs pour une app mobile.
				\item \textbf{outils}: C++, Python, Sklearn, StatsModels, Seaborn
    \end{itemize}} \\
		
     \twentyitem
   		{2015 - 2015}
		{}
        {Data Scientist}
        {\href{http://www.keyrus.com/}{Keyrus}}
        {}
        {
        \begin{itemize}
        \item D\'eveloppement d'une plateforme de machine learning sur AWS. 
				\item \textbf{outils}: Spark, MLlib, Hbase, Cassandra
    \end{itemize} }\\
		
	\twentyitem
   		{2011 - 2013}
		{}
        {Enseignant-Chercheur ATER}
        {\href{http://www.univ-lyon1.fr/}{Universit\'e Lyon I/ Montpellier II}}
        {}
        {}
        
	%\twentyitem{<dates>}{<title>}{<location>}{<description>}
\end{twenty}





%----------------------------------------------------------------------------------------
%	 EDUCATION
%----------------------------------------------------------------------------------------
\vspace{-0.5cm}
\section{Formation}{\faGraduationCap}

\begin{twenty} % Environment for a list with descriptions
	\twentyitemshorttest
    	{2014 - 2015}
        {}
        {Mast\`ere, Big Data/Machine Learning}
        {Telecom ParisTech}{}
				
	\twentyitemshorttest
    	{2008 - 2012}
		{}
        {Ph.D en Math\'ematiques}
        {\href{http://www.umontpellier.fr/}{Universit\'e Montpellier II}}
        {}
				
	\twentyitemshorttest
    	{2006 - 2008}
		{}
        {MSc. en Math\'ematiques et Statistiques}
        {\href{http://www.umontpellier.fr/}{Universit\'e Montpellier II}}
        {Major, Mention TB}
	%\twentyitem{<dates>}{<title>}{<organization>}{<location>}{<description>}
\end{twenty}

\section{Projets - Recherche}{\faClipboard}
\begin{twenty}
	\twentyitem
			{2018 - }
			{Pr\'esent}
			{Contribution Open source Scikit-Learn}
			{Github - sklearn}
			{}
			{}
	\twentyitem
			{2015 - }
			{Pr\'esent}
			{Data Science, Machine Learning Challenge}
			{Kaggle - DataScience}
			{}
			{\begin{itemize}
			 \item Quora - Cdiscount - Avito - Human or Robot - SpringLeaf
			 \item \textbf{outils}: Python, Xgboost, TensorFlow, Keras, NLTK, gensim
			\end{itemize}}
	\twentyitem
    	{2014 - 2015}
		{}
        {Mast\`ere, Big data/Machine Learning}
        {Yuzu}
        {}
        {\begin{itemize}
        \item \'Elaboration d'un syst\`eme scalable de recommandations.
        \item Cross-validation du syst\`eme en prenant en compte la structure temporelle.
        \item \textbf{tools}: Spark, MLlib, Zeppelin, Elastic-Search
		\end{itemize}}
	\twentyitem
			{2008 - 2012}
			{}
			{Ph.D en Math\'ematiques}
			{\href{http://www.umontpellier.fr/}{University of Montpellier II}}
			{}
			{\begin{itemize}
			\item \textbf{th\`ese}: Quantization/Deformation de sous-alg\`ebres de Lie coisotropes.
			\item \textbf{mots cl\'es}: Quantum Groups, Universal quantization, Lie Bialgebra.
			\end{itemize}}
\end{twenty}
\end{document} 
